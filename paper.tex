\documentclass[twocolumn,11pt]{article}

\usepackage{amssymb}
\usepackage{amsmath}
\usepackage{graphicx}
\usepackage{epstopdf}
\usepackage{fancyhdr}
\usepackage{algpseudocode}
\usepackage[toc,page]{appendix}
\usepackage{url}

\setlength{\parskip}{5pt plus1pt minus1pt}

\DeclareMathSizes{36}{36}{36}{36}

\title{Train Harder, Better:\\Using graph theory to improve your
  training regimen}
\author{Forest Trimble\\trimbf@rpi.edu}
\begin{document}

\pagestyle{fancy}
\fancyhead{}
\fancyhead[L]{Forest Trimble}
\fancyhead[R]{Train Harder, Better}
\maketitle

\begin{abstract}
  \emph{Cyclists are always in search of the perfect ride on the perfect
  roads. They have criteria like distance and elevation gain to ensure
  that they get in the workout that they want. Unfortunately, in order to
  find this ideal ride manually, it takes a great deal of exploring and
  time, and eventually, one may settle into the habit of using the same
  roads that he/she already knows. We research a way to improve this
  paradigm, and to generate cyclists exactly the route they are looking
  for, without them having to do any work.}
\end{abstract}

\section{Background}

Cycling can be wildly different based on the roads that one takes: On busy
roads with no shoulders, it can be borderline miserable, while few things in
the world are better than spinning down a smoothly paved road with beautiful
vistas of open countryside. Unfortunately, cyclists need to invest massive
amounts of time and energy exploring the roads and amassing a repertoire that
they can use. Additionally, it is difficult to satisfy criteria for training,
like an elevation gain and distance, using only that mental repertoire. This
paper chronicles the attempt to unroll a solution for this problem.

Specifically, a good solution should take as input a distance to travel, a
start point, and, optionally, an elevation gain, and generate a route from the
start point that satisfies the distance and elevation gain within some
$\epsilon$. A fully robust solution would also utilize user data to ensure that
it traverses the most pleasant roads to ride on and avoids the worst.

\section{Tools}

In order to do this in a useful way

\end{document}
